
\documentclass[../ana2.tex]{subfiles}
\begin{document}
\setcounter{section}{8}
\section{Topologie des \( \R^d (\C^d, \ldots) \)}
Frage: Abstand, Konvergenz in \gqq{komplizierteren} 
Räumen.
Z. B. 
\[ \R^d = \set{x = (x_1, \ldots, x_d) : 
x_j \in \R \forall 1 \leq j \leq d} \]
\[ \C^d = \set{z = (z_1, \ldots, z_d) : 
z_j \in \R \forall 1 \leq j \leq d} \]
\begin{defi}[Norm auf einen Vektorraum]
    Eine Norm auf enen reellen oder komplexen Vektorraum
    ist eine Funktion: \( \abb{\norm{\cdot}}{X}{\R} \) mit
    \begin{description}
        \item[Positivität] \( \norm{x} \geq 0 \forall x\in X \) 
        und \( \norm{x} = 0 \Leftrightarrow x = 0 \).
        \item[Homogenität] \( \norm{\lambda x} = \abs{\lambda} \norm{x}
        \; \forall \lambda \in \R (\C), x\in X \).
        \item[\( \Delta \)-Ungleichung] \( \norm{x+y}
        \leq \norm{x} + \norm{y} \; \forall x,y \in X \)
    \end{description}
\end{defi}
\begin{bspe}
    \[ \norm{x}_1 = \sum_{j=1}^d \abs{x_j} \]
    \[ \norm{x}_\infty 
    = \underset{1 \leq j \leq d}{\max} \abs{x_j} \]
\end{bspe}
\begin{defi}[Metrik und metrischer Raum]
    Menge \(X\) mit einer Funktion \( \abb{d}{X\times X}{\R} \)
    mit den Eigenschaften
    \begin{description}
        \item[Positivität] \( d(x,y) \geq 0, d(x,y) = 0 \Leftrightarrow x=y
        \; \forall x,y \in X \)       
        \item[Symmetrie] \( d(x,y) = d(y,x) \; \forall x,y \in X \)
        \item[\( \Delta \)-Ungleichung] 
        \( d(x,y) \leq d(x,z) + d(z, y) 
        \; \forall x,y,z \in X \).
    \end{description}
    \(d\) heißt Metrik und \( (X, d) \) heißt metrischer 
    Raum.
\end{defi}
\begin{bsp}
    \( \R^d \) oder \( \C^d\) mit \(\norm{\cdot}_2\)
    \[ d(x,y) := \norm{x-y}_2 \]
    oder ein allgemeiner Vektorraum \(X\) mit Norm 
    \(\norm{\cdot}\)
    \[ d(x,y) := \norm{x-y} \]
    Z. B. 
    \begin{align*}
        d(y,x) &= \norm{y - x} = \norm{-(x-y)} \\
        &= \abs{-1}\norm{x-y} = \norm{x-y} = d(x, y).
    \end{align*}
    \begin{align*}
        d(x,y) &= \norm{x-y} = \norm{x-z + (z-y)} \\
        &\leq \norm{x-z} + \norm{z-y} \\
        &= d(x,z) + d(z,y)
    \end{align*}
    \( \R^d \) mit \( \norm{\cdot}_2 \) ist ein metrischer 
    Raum.
\end{bsp}
\begin{defi}
    Sei \(X\) ein metrischer Raum mit Metrik \(d\).
    \[ x_0 \in X, r > 0 \quad B_r(x_0) 
    := \set{ x\in X: d(x, x_0) < r } \] 
    ist die offnee Kugel mit Radius \(r\) um \(x_0\).
    \[ \R^d: B_r(x_0) = \set{x \in \R^d: 
    \norm{x - x_0}_2 < r}. \]
\end{defi}
\begin{defi}[offene Mengen]
    Sei \( (X,d) \) ein metrischer Raum. Eine Menge 
    \( U \subset X \) heißt offen, falls 
    \[ \forall x\in U \exists r = r_x > 0 \]
    mit 
    \[ B_r(x) \subset U. \]
\end{defi}
\begin{bsp}
    \( B_r(x_0) \) ist offen: 
    \( x \in B_r(x_0) \).
    \( \varepsilon := r - d(x, x_0) \)
    Ist \( y \in B_\varepsilon(x): d(y, x_0) 
    \leq d(y, x) + d(x, x_0) < \varepsilon 
    + d(x, x_0) = r \).
\end{bsp}
\begin{satz}[Topologie]
    Man nehme metrischen Raum \( (X, d) \).
    Die Menge aller offenen Teilmengen von \(X\) 
    ist eine Topologie, das heißt
    \begin{enumerate}[label=(\alph*)]
        \item \( \emptyset, X \) sind offen.
        \item Der Durchschnitt endlich vieler
        offener Mengen ist offen.
        \item Die Vereinigung beliebig vielen offenen
        Mengen ist offen. 
    \end{enumerate}    
\end{satz}
\begin{bew}
    Etwas später.
\end{bew}
\begin{defi}
    Sei \( X, d \) ein metrischer Raum, \( M \subset X \).
    Dann ist 
    \[ \inner M = M^o 
    := \set{x \in M: \exists r_x > 0: B_r(x \subset M)}\]
    \[= \set{x \in M: \exists \varepsilon > 0: 
    B_\varepsilon(x) \subset M }\]
    Abschluss:
    \[ \overline{M} := 
    \set{x \in M: \; \forall \varepsilon > 0: 
    B_\varepsilon(x) \cap M \neq \emptyset} \]
    Rand: 
    \[ \delta M := \set{ x \in X: \forall \varepsilon > 0: 
    B_\varepsilon(x) \cap M \neq \emptyset, 
    B_\varepsilon(x) \cap M^C \neq \emptyset },
    M^C = X \setminus M. \]
    \[ \inner M \subset M \subset \overline{M}. \]
\end{defi}
\end{document}