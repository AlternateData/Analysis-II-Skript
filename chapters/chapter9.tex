
\documentclass[../ana2.tex]{subfiles}
\begin{document}
\setcounter{section}{8}
\section{Topologie des \( \R^d (\C^d, \ldots) \)}
Frage: Abstand, Konvergenz in \gqq{komplizierteren} 
Räumen.
Z. B. 
\[ \R^d = \set{x = (x_1, \ldots, x_d) : 
x_j \in \R \forall 1 \leq j \leq d} \]
\[ \C^d = \set{z = (z_1, \ldots, z_d) : 
z_j \in \R \forall 1 \leq j \leq d} \]
\begin{defi}[Norm auf einen Vektorraum]
    Eine Norm auf enen reellen oder komplexen Vektorraum
    ist eine Funktion: \( \abb{\norm{\cdot}}{X}{\R} \) mit
    \begin{description}
        \item[Positivität] \( \norm{x} \geq 0 \forall x\in X \) 
        und \( \norm{x} = 0 \Leftrightarrow x = 0 \).
        \item[Homogenität] \( \norm{\lambda x} = \abs{\lambda} \norm{x}
        \; \forall \lambda \in \R (\C), x\in X \).
        \item[\( \Delta \)-Ungleichung] \( \norm{x+y}
        \leq \norm{x} + \norm{y} \; \forall x,y \in X \)
    \end{description}
\end{defi}
\begin{bspe}
    \[ \norm{x}_1 = \sum_{j=1}^d \abs{x_j} \]
    \[ \norm{x}_\infty 
    = \underset{1 \leq j \leq d}{\max} \abs{x_j} \]
\end{bspe}
\begin{defi}[Metrik und metrischer Raum]
    Menge \(X\) mit einer Funktion \( \abb{d}{X\times X}{\R} \)
    mit den Eigenschaften
    \begin{description}
        \item[Positivität] \( d(x,y) \geq 0, d(x,y) = 0 \Leftrightarrow x=y
        \; \forall x,y \in X \)       
        \item[Symmetrie] \( d(x,y) = d(y,x) \; \forall x,y \in X \)
        \item[\( \Delta \)-Ungleichung] 
        \( d(x,y) \leq d(x,z) + d(z, y) 
        \; \forall x,y,z \in X \).
    \end{description}
    \(d\) heißt Metrik und \( (X, d) \) heißt metrischer 
    Raum.
\end{defi}
\begin{bsp}
    \( \R^d \) oder \( \C^d\) mit \(\norm{\cdot}_2\)
    \[ d(x,y) := \norm{x-y}_2 \]
    oder ein allgemeiner Vektorraum \(X\) mit Norm 
    \(\norm{\cdot}\)
    \[ d(x,y) := \norm{x-y} \]
    Z. B. 
    \begin{align*}
        d(y,x) &= \norm{y - x} = \norm{-(x-y)} \\
        &= \abs{-1}\norm{x-y} = \norm{x-y} = d(x, y).
    \end{align*}
    \begin{align*}
        d(x,y) &= \norm{x-y} = \norm{x-z + (z-y)} \\
        &\leq \norm{x-z} + \norm{z-y} \\
        &= d(x,z) + d(z,y)
    \end{align*}
    \( \R^d \) mit \( \norm{\cdot}_2 \) ist ein metrischer 
    Raum.
\end{bsp}
\begin{bsp}[beschränkte Funktionen von \( X \) nach \( \R \)]
    \[ L^\infty(x) := \set{\abb{f}{X}{\R} 
    \; \vert \; \underset{x\in X}{\sup} \abs{f(x)} < \infty} \]
    \[ d_\infty(f, g) := \underset{x\in X}{\sup} 
    \abs{ f(x) - g(x) } = \norm{f-g}_\infty \]
    Angenommen: \(X\) ist ein metrischer Raum, 
    \( \abb{d}{X \times X}{\R} \). \\
    Nehme \( a \in X \), setze 
    \( f_x(y) := -d(y,a) + d(x,y), y \in X \) \\
    \( \Rightarrow \) Die Abbildung \( X \rightarrow L^\infty \)
    \( x \mapsto f_x \) (\( f_x(y) := 
    -d(y,a) + d(x,y), a \in X \))
    ist injektiv.
\end{bsp}
\begin{bsp}
    Norm \( \norm{\cdot} \) auf VR \( V, 
    d(x,y) := \norm{x-y} \) ist Metrik auf 
    \(V\).
\end{bsp}
\begin{defi}
    Sei \(X\) ein metrischer Raum mit Metrik \(d\).
    \[ x_0 \in X, r > 0 \quad B_r(x_0) 
    := \set{ x\in X: d(x, x_0) < r } \] 
    ist die offnee Kugel mit Radius \(r\) um \(x_0\).
    \[ \R^d: B_r(x_0) = \set{x \in \R^d: 
    \norm{x - x_0}_2 < r}. \]
\end{defi}
\begin{defi}[offene Mengen]
    Sei \( (X,d) \) ein metrischer Raum. Eine Menge 
    \( U \subset X \) heißt offen, falls 
    \[ \forall x\in U \exists r = r_x > 0 \]
    mit 
    \[ B_r(x) \subset U. \]
\end{defi}
\begin{bsp}
    \( B_r(x_0) \) ist offen: 
    \( x \in B_r(x_0) \).
    \( \varepsilon := r - d(x, x_0) \)
    Ist \( y \in B_\varepsilon(x): d(y, x_0) 
    \leq d(y, x) + d(x, x_0) < \varepsilon 
    + d(x, x_0) = r \).
\end{bsp}
\begin{satz}[Topologie]
    Man nehme metrischen Raum \( (X, d) \).
    Die Menge aller offenen Teilmengen von \(X\) 
    ist eine Topologie, das heißt
    \begin{enumerate}[label=(\alph*)]
        \item \( \emptyset, X \) sind offen.
        \item Der Durchschnitt endlich vieler
        offener Mengen ist offen.
        \item Die Vereinigung beliebig vielen offenen
        Mengen ist offen. 
    \end{enumerate}    
\end{satz}
\begin{bew}
    Etwas später.
\end{bew}
\begin{defi}
    Sei \( (X, d) \) ein metrischer Raum, \( M \subset X \).
    Dann ist 
    \[ \inner M = M^o 
    := \set{x \in M: \exists r_x > 0: B_r(x)\subset M}\]
    \[= \set{x \in M: \exists \varepsilon > 0: 
    B_\varepsilon(x) \subset M }\]
    Abschluss:
    \[ \overline{M} := 
    \set{x \in X: \; \forall \varepsilon > 0: 
    B_\varepsilon(x) \cap M \neq \emptyset} \]
    Rand: 
    \[ \delta M := \set{ x \in X: \; \forall \varepsilon > 0: 
    B_\varepsilon(x) \cap M \neq \emptyset, 
    B_\varepsilon(x) \cap M^C \neq \emptyset },
    M^C = X \setminus M. \]
    \[ \inner M \subset M \subset \overline{M}. \]
\end{defi}
\begin{lem}[Hausdorff-Trennungseigenschaft]
    Sei \( X \) ein metrischer Raum, 
    \( x,y \in X, x\neq y \). 
    \[ \Rightarrow \exists \varepsilon > 0: 
    B_\varepsilon(x) \cap B_\varepsilon(y) = \emptyset \]
\end{lem}
\begin{bew}
    Ist \(x \neq y \Rightarrow L = d(x,y) > 0 \) \\
    Sei \( \varepsilon := \frac{L}{2} \) und 
    \( B_\varepsilon(x) := \set{z \in X: d(z,x) < \varepsilon} \)
    Angenommen \( z \in B_\varepsilon(x) \cap 
    B_\varepsilon(y) \)
    \[ \Rightarrow L = d(x,y) 
    \leq d(x,z) + d(z,y) 
    \leq \varepsilon + \varepsilon = 2\varepsilon = L. \]
\end{bew}
\begin{defi}[Konvergenz]
    Sei \( X \) ein metrischer Raum. Die Folge 
    \( (x_n)_n \) von Punkten \( x_n \in X \; \forall n\in\N \)
    (schreiben \( (x_n)_n \subset X \)) konvergiert 
    gegen \( x \in X \), falls 
    \begin{align*}
        &\forall \varepsilon > 0 \; \exists K \in \N: 
        d(x_n, x) < \varepsilon \;\forall n \geq K. \\
        &\Leftrightarrow \; \forall \varepsilon > 0 
        \; \exists K \in \N: x_n \in B_\varepsilon(x) 
        \; \forall n \geq K 
    \end{align*}
    \( \Leftrightarrow \forall \varepsilon > 0  \) 
    ist \( x_n \in B_\varepsilon(x) \) für fast alle 
    \( n\in\N \)
    \[ \Leftrightarrow \limes{n} d(x_n, x) = 0 \]
    Schreiben \( x_n \rightarrow x \), oder 
    \( \limes{n} x_n = x \).
\end{defi}
\begin{bem}
    Der Grenzwert \( x \) ist eindeutig. (wg. Hausdorff)
\end{bem}
\begin{defi}[Abgeschlossene Menge]
    Sei \( X \) ein metrischer Raum und \( A \subset X \) 
    heißt abgeschlossen, falls 
    \[ (x_n)_n \subset A, x_n \rightarrow x 
    \Rightarrow x \in A. \]
\end{defi}
\begin{bem}
    \(\emptyset, X \) sind abgeschlossen.
\end{bem}
\begin{satz}
    In einem metrischen Raum 
    \( X \) gilt für alle \( A \subset X \):
    \[ A \text{ ist offen } 
    \Leftrightarrow A^C = X \setminus A \text{ ist 
    abgeschlossen}. \]
\end{satz}
\begin{bew}
    \gqq{\( \Rightarrow \)}: Sei \(M\) offen.
    \( M^C = X \setminus M \). \( (x_n)_n 
    \in M^C, x_n \rightarrow x \).

    Angenommen \( x \notin M^C\), d. h. \(x \in M \).
    \(M\) offen \(\Rightarrow \exists \varepsilon > 0:
    B_\varepsilon(x) \subset M. \) \\
    \( \Rightarrow \) ist \(y \in M^C: d(y, x) 
    \geq \varepsilon \) \\
    Andererseits \( d(x_n, x) \rightarrow 0 
    (n \rightarrow \infty)  \) \Lightning{} \\
    \( \Rightarrow x \in M^C \)
    \gqq{\( \Leftarrow \)}:
    Annahme: \( M^C \) ist abgeschlossen\\
    Angenommen, \(M\) wäre nicht offen.
    \[ \Rightarrow \exists x\in M: 
    \forall \varepsilon > 0 : B_\varepsilon(x) \cap 
    M^C \neq \emptyset. \]    
    Wähle \( \varepsilon = \frac{1}{n} 
    \Rightarrow \exists x_n \in B_{\frac{1}{n}}(x) \cap 
    M^C \).
    \begin{align*}
        &\Rightarrow (x_n)_n \subset M^C, x_n \rightarrow x 
        \overunderset{M^C}{\text{abgeschlossen}}{\Rightarrow} 
        x \in M^C \\
        &\Rightarrow x \in M \cap M^C = \emptyset \text{ \Lightning} 
    \end{align*}
\end{bew}
\begin{kor}
    Für Teilmengen eines metrischen Raums \(X\) gilt: 
    \begin{enumerate}
        \item \( \emptyset, X \) abgeschlossen.
        \item Die Vereinigung endlich vieler abgeschlossen 
        Mengen ist abgeschlossen.
        \item Der Durchschnitt beliebig vieler abgeschlossener 
        Mengen ist abgeschlossen.
    \end{enumerate}
\end{kor}
\begin{bew}
    Satz 10 und de Morgan Regeln.
\end{bew}
\begin{bem}
    \( I \) Indexmenge, \( j \in I \) ist 
    \( U_j \subset X \) offen.
    \[ U = \bigcup_{j\in I} U_j \]
    \[ x \in U \Leftrightarrow \exists j_0 \in I: 
    x \in U_{j_0}. \]
\end{bem}
\begin{satz}
    Sei \( M \) Teilmenge eines metrischen 
    Raumes \( X \).
    \begin{enumerate}
        \item \( \inner M \) ist offen und \( U \) offen 
        und \( U \subset M \Rightarrow U \subset \inner M \).
        (d. h. \( \inner M \) ist die größte offene Teilmenge 
        von \(M\))
        \item \( \overline{M} \) ist abgeschlossen und ist 
        \(A \supset M \), \( A \) abgeschlossen 
        \[ \Rightarrow A \supset \overline{M}. \]
        \[ \text{d. h. } \overline{M} 
        = \bigcap_{\substack{A \supset M\\A \text{ abg.}}} 
        U \subset M \]
        \item \( \delta M = \overline{M} \setminus (\inner M) \).
    \end{enumerate}
\end{satz}
\begin{bew}
    1.\\
    \( x \in \inner M \). Das heißt 
    \[ \exists r > 0: B_r(x) \subset M. \]
    Sei \( y \in B_r(x) \) und \( \varepsilon := d(y,x) > 0 \).
    \[ \Rightarrow B_\varepsilon(y) \subset B_r(x) 
    \subset M. \]
    \[ \Rightarrow y \in \inner M \]
    \[ \Rightarrow B_r(x) \subset \inner M \]
    \[ \Rightarrow \inner M \text{ ist offen}. \]
    Sei \( U \subset M \) offen. 
    \( \Rightarrow \forall x \in U 
    \exists \varepsilon = \varepsilon_x > 0: 
    B_\varepsilon(x) \subset U \subset M \).

\end{bew}
\begin{satz*}\leavevmode
    \begin{enumerate}
        \item \( \inner M \) offen und 
        \( U \subset M \) offen \( \Rightarrow U \subset \inner M \)
        (d. h. \( \inner M = \bigcup_{\substack{U \subset M \\\text{offen}}} \) 
        größte offene Teilmenge von \(M\))
        \item \( \overline{M} \) ist abgeschlossen und 
        \( M \subset A \) abgeschlossen 
        \( \Rightarrow \overline{M} \subset A \)
        \item \( \delta M = \overline{M} \setminus \inner M \)
    \end{enumerate}
\end{satz*}
\begin{bew}
    zu 2. \( \overline{M} = \set{x \in X: \forall \varepsilon > 0 
    \text{ ist } B_\varepsilon(x) \cap M \neq \emptyset} \) \\
    \( \overline{M}^C = \set{x \in X: \exists \varepsilon > 0: 
    B_\varepsilon(x) \subset M^C} = \inner M^C \) \\
    \( \overundersett{Satz}{10}{\Rightarrow} 
    \overline{M} \) ist abgeschlossen.\\
    Ist \(A \supset M, A\) abgeschlossen 
    \( \overundersett{Satz}{10}{\Rightarrow}
    A^C \) ist offen und 
    \( A^C = X \setminus A \subset X \setminus M = M^C
    \Rightarrow A^C \subset \inner M^C 
    = \overline{M}^C \Rightarrow A \supset \overline{M} \)
    Zu 3. Nach Definition ist 
    \[ \delta M = \set{x \in X : 
    \forall \varepsilon > 0 \text{ ist } 
    B_\varepsilon(x) \cap M \neq \emptyset 
    \text{ und } B_\varepsilon(x) \cap M^C \neq \emptyset} \]
    \[ = \overline{M} \cap \overline{M^C} 
    = \overline{M} \cap (m + M)^C 
    = \overline{M} \setminus (\inner M). \]
    Da \[ (m+M)^C = \set{x \in X: \exists \varepsilon > 0: 
    B_\varepsilon(x) \subset M}^C \]
    \[ = \set{ x \in X: \forall \varepsilon > 0: 
    B_\varepsilon(x) \cap M^C \neq \emptyset } \]
    \[ = \overline{M^C}. \]
\end{bew}
\begin{defi}[Häufungspunkte, isolierte Punkte]
    Set \(X\) metrischer Raum, \( M \subset X \) \\
    Ein Punkt \( x \in X \) heißt Häufungspunkt von \(M\),
    falls 
    \[ \forall \varepsilon > 0: 
    \inner (B_\varepsilon(x) \cap M) \setminus \set{x} 
    \neq \emptyset. \]
    Ein Punkt \( x \in X \) heißt isolierter Punkt 
    von \( M \), falls 
    \[ \exists \varepsilon > 0: B_\varepsilon(x) \cap M 
    = \set{x}. \]
\end{defi}
\begin{bem}
    Ist \( x \) ein Häufungspunkt von \( M \), 
    so enthält \( B_\varepsilon(x) \cap M \) 
    sogar \( \infty \)-viele Puntke auf \( M \) 
    (\( \forall \varepsilon > 0 \)).
\end{bem}
\begin{bew}
    Angenommen, es wäre falsch.
    \( \exists \varepsilon_0 > 0: 
    (B_{\varepsilon_0}(x) \setminus \set{x}) \cap M \)
    enthält endlich viele Puntke \( y_1, \ldots, y_k \).
    \[ \varepsilon_\delta = d(y_i, x) > 0 \]
    \[ \varepsilon := 
    \min(\varepsilon_1, \dots, \varepsilon_k) > 0 \]
    \[ \Rightarrow (B_\varepsilon(x) \setminus \set{x})
    \cap M = \emptyset. \text{ \Lightning{}} \]
\end{bew}
\begin{defi}
    Eine Teilmenge \(M\) eines metrischen Raumes heißt
    dicht, falls \(\overline{M} = X\).\\
    z. B. \( \Q \) ist dicht in \( \R \), 
    \( \Q^d \) ist dicht in \( \R^d \).
\end{defi}
\begin{defi}[Cauchyfolge]
    Sei \( X \) ein metrischer Raum. 
    Eine Folge \( (x_n)_n \subset X \) heißt 
    Cauchyfolge, falls
    \[ \forall \varepsilon > 0 \; \exists K \in \N: 
    d(x_n, x_m) < \varepsilon \; \forall n,m \geq K \]
    \[ \Leftrightarrow \limessup{n} \limessup{m} 
    d(x_n, x_m) = 0. \]
    Ein metrischer Raum heißt vollständig, falls jede 
    Cauchyfolge konvergiert.
\end{defi}
\begin{bem}\leavevmode
    \begin{enumerate}
        \item Ist \( M \) ein metrischer Raum 
        und \( (x_n)_n \) konvergiert in \( X \)
        (gegen ein \( x \in M \)) 
        \( \Rightarrow (x_n)_n \) ist eine 
        Cauchyfolge.
        \item Bsp.: \( \R^d \) mit \( \norm{x}_2 = \abs{x} 
        = \left( \sum_{j=1}^d \abs{x_j}^2 \right)^2) \)
        ist vollständig. \\
        \( \C^d \) mit \( \norm{z}_2 = \abs{z} 
        = \left( \sum_{j=1}^d \abs{z_j}^2 \right)^2) \)
        ist vollständig.
        \item \( X \) eine Menge, 
        \( L^\infty(X) = L^\infty(X, \R) 
        = \set{ \abb{f}{X}{\R}, 
        \norm{f}_\infty 
        := \underset{x \in X}{\sup} \abs{f(x)} < \infty } \)
        \[ L^\infty(X, \C) 
        = \set{ \abb{f}{X}{\C}, 
        \norm{f}_\infty 
        := \underset{x \in X}{\sup} \abs{f(x)} < \infty } \]
        ist vollständig, \( d(f,g) = d_\infty(f,g) 
        := \norm{f-g}_\infty = \norm{f-g}_\infty 
        := \underset{x\in X}{\sup} \abs{f(x) - g(x)} \)
    \end{enumerate}
\end{bem}
\begin{bew}
    Sei \( (f_n)_n \) eine Cauchyfolge in 
    \( L^\infty(X, \R) \).
    Schritt 1: Kandidat: \\
    Sei \( x \in X \), \( (f_n(x))_n \) reelle Folge.
    Ist sogar eine Cauchyfolge in \(\R\).
    \[ \abs{f_n(x) - f_m(x)} \leq \underset{y \in X}{\sup} 
    \abs{f_n(y)-f_m(y)} = \norm{f_n-f_m} \]
    \[ \limessup{n} \limessup{m} \norm{f_m - f_n}_\infty 
    = 0 \]
    \[ \Rightarrow \limessup{n} \limessup{m} 
    \abs{f_n(x)-f_m(x)} = 0 \]
    \( \overunderset{\R}{\text{vollständig}}{\Rightarrow} 
    f(x) := \limes{n} f_n(x)\) existiert. \\
    Brauchen noch: 
    \begin{enumerate}
        \item \( f \in L^\infty(X, \R) \)
        \item \(f_n\) konvergiert gegen \(f\) in 
        \( L^\infty(X, \R) \)
    \end{enumerate}
    Schritt 2: \(f \in L^\infty(X, \R)\) \\
    \( f_n \) ist Cauchyfolge in \(L^\infty(X,\R)\)
    d. h. \( \exists C < \infty: \norm{f_n}_\infty 
    \leq C \; \forall n \in \N \) \\
    Demnach \( \forall \varepsilon > 0 \; \exists K \in \N: 
    \norm{f_n-f_m}_\infty < \varepsilon \; \forall n,m > K \) \\
    nehme \( \varepsilon = 1 \Rightarrow \; \exists K: \norm{f_n-f_m} 
    < 1 \; \forall n,m \geq K \)

    \( \Rightarrow \) ist \( n \geq K: 
    \norm{f_n - f_K}_\infty < 1 \).
    \[ \norm{f_n}_\infty \leq 
    \norm{ f_n - f_K }_\infty 
    + \norm{f_K}_\infty < 1 + \norm{f_K}_\infty \]
    \[ C := \max( \norm{f_1}_\infty, 
    \norm{f_2}_\infty, \ldots, 
    \norm{f_K}_\infty, 1 + \norm{f_K}_\infty ) \]
    \[ \Rightarrow \norm{f_n}_\infty \leq C \;\forall \; 
    n\in\N. \]
    \(x \in X\):
    \[ \abs{f(x)} = \limes{n} \abs{f_n(x)} 
    = \limesinf{n} \abs{f_n(x)}
    \leq \limesinf{n} \norm{f_n}_\infty \]
    mit \( \abs{f_n(x)} 
    \leq \underset{y \in X}{\sup} \abs{f_n(y)} 
    = \norm{f_n}_\infty \)
    \[ \Rightarrow \norm{f}_\infty 
    = \underset{x \in X}{\sup} \abs{f(x)}
    \leq \limesinf{n} \norm{f_n}_\infty \]
    Schritt 3:
    \[ f(x) - f_n(x) = 
    \limes{m} \abs{ f_m(x) - f_n(x) }
    = \limessup{m} \underbrace{\abs{f_m(x) - f_n(x)}}
    _{\leq \norm{f_m - f_n}_\infty} \]
    \[ \leq \limessup{m} \norm{f_m - f_n}_\infty. \]
    \[ \Rightarrow \norm{f - f_n}_\infty 
    = \underset{x \in X}{\sup} \abs{f(x) - f_n(x)} 
    \leq \limessup{m} \norm{f_m - f_n}_\infty. \]
    \[ \Rightarrow \limessup{n} \norm{f-f_n}_\infty 
    \leq \limessup{n}\limessup{m} 
    \norm{f_m - f_n}_\infty = 0. \]
\end{bew}
\begin{beh}
    \( L^\infty(\set{1,2,\dots,d}, \R) \cong \R^d \)
\end{beh}
\begin{bew}
    Sei \( x = (x_1,\dots,x_d) \in \R^d \) \\
    \( \Rightarrow \) Funktion 
    \( \abb{h_x}{\set{1,\dots,d}}{\R}, j \mapsto h_x(j) := x_j \)
    Umgekehrt ist \( h \in L^\infty(\set{1,\dots,d}, \R) \)
    mit \( x_j := h(j) \). \\
    Normen passen:
    \[ \norm{x}_\infty = \underset{j=1,\dots,d}{\max} \abs{x_j}
    = \underset{j=1,\dots,d}{\max} \abs{h_x(j)} = \norm{h_x}_\infty \]
    Genauso \( L^\infty(\set{1,\dots,d}, \C) \cong \C^d \)
    Warum ist \( \R^d, \norm{\cdot}_2 \) vollständig?
    Z. B. \( \norm{\cdot}_2 \) und \( \norm{\cdot}_\infty \) 
    sind äquivalent auf \( \R^d \).
    \[ \norm{x}_\infty 
    = (\underset{j=1,\ldots,d}{\max} 
    \abs{x_j}^2 )^{\frac{1}{2}} 
    \leq (\sum_{j=1}^d \abs{x_j}^2)^{\frac{1}{2}}
    = \norm{x}_\infty 
    \leq (\sum_{j=1}^d \norm{x}_\infty^2)^{\frac{1}{2}}
    = \sqrt{d} \norm{x}_\infty. \]
\end{bew}
\begin{defi*}[Äquivalenz von Normen]
    \( \norm{:}_a, \norm{.}_b \) sind äquivalent, wenn
    \[ \exists c_1, c_2: c_1\norm{x}_a \leq \norm{x}_b 
    \leq c_2\norm{x}_a \; \forall x \]
\end{defi*}
\begin{bsp}
    \( \mathcal{C}^\infty(\N) 
    = \set{\text{reelle oder komplexe Folgen}
    (x_n)_n, \underset{n\in\N}{\sup} \abs{x_n} < \infty} \) ist
    vollständig.
    \( \mathcal{C}^\infty(\N) 
    = \set{\abb{h}{\N}{\R}, \norm{h}_\infty < \infty}
    = L^\infty(\N, \R) \)
\end{bsp}
\begin{defi}[Stetigkeit]
    Seien \(X, Y \) metrische Räume, \(\abb{f}{X}{Y}\)
    heißt stetig in \(x_0 \in X\), falls 
    \[ \forall \varepsilon > 0 \; \exists \delta > 0
    d_y(f(x), f(x_0)) < \varepsilon \forall x \in X: 
    d_x(x, x_0) < \delta \]
    oder äquivalent 
    \[ f(B_\delta(x_0)) \subset B_\varepsilon(f(x_0)) \]
    \(f\) stetig auf \(X\) falls \(f\) in jedem Punkt 
    \(x_0 \in X\) stetig ist. \\
    Schreiben \( d(f(x), f(x_0)) \) 
    statt \( d_y(f(x), f(x_0)) \) usw.
\end{defi}    
\begin{defi}[Lipschitzstetig]
    \( \abb{f}{X}{Y} \) heißt Lipschitzstetig 
    mit Konstante \( L \geq 0 \), falls 
    \[ d(f(x_1), f(x_2)) 
    \leq L d(x_1, x_2) \forall x_1, x_2 \in X. \]
\end{defi}
\begin{bsp}
    Sei \(X\) ein metrischer Raum, \(x_0 \in X\),
    \( \abb{f}{X}{[0, \infty)}: f(x) := d(x,x_0) \)
    \[ f(x_1) = d(x_1, x_0) \leq d(x_1, x_2) + d(x_2, x_0)
    = d(x_1, x_2) + f(x_2) \]
    \[ \Rightarrow f(x_1) - f(x_2) \leq d(x_1, x_2) \]
    \[ \Rightarrow \underbrace{f(x_2) - f(x_1)}_{
        = -(f(x_1) - f(x_2))
    } \leq d(x_2, x_1) = d(x_1, x_2) \]
    \[ \Rightarrow \abs{ f(x_1) - f(x_2) } 
    \leq d(x_1, x_2). \]
\end{bsp}
\begin{defi}[Folgenkriterium für Stetigkeit]
    Sei \( \abb{f}{X}{Y} \) stetig in \(x_0\)\\
    \( \Leftrightarrow \) für jede Folge 
    \( (x_n)_n \subset X, x_n \rightarrow x_0 \) 
    folgt \( f(x_n) \rightarrow f(x_0) \).
\end{defi}
\begin{bew}
    \( \Rightarrow \): 
    \[ \forall \varepsilon >0
    \exists \delta >0: d(f(x), f(x_0)) < \varepsilon
    \forall x \in X: d(x, x_0) < \delta \tag{\(*\)} \]
    Nehme \( (x_n)_n \subset X, x_n \rightarrow x_0 \).
    Nehme \( \varepsilon > 0, \delta > 0 \), sodass 
    \((*)\) gilt.
    \( \Rightarrow d(x_n, x_0) < \delta \) für fast 
    alle \(n\).\\
    \( \Rightarrow d(f(x_n), f(x_0)) < \varepsilon \) 
    für fast alle \(n\).\\
    \( \rightarrow f(x_n) \rightarrow f(x_0) \). \\
    \( \Leftarrow \): Kontraposition \\
    Angenommen \(f\) ist nicht stetig in \(x_0\)
    \( \Rightarrow \exists \varepsilon_0 > 0: 
    \forall \delta >0: \exists x\in X: 
    d(f(x),f(_0)) \geq \varepsilon_0, d(x,x_0) < \delta \) \\
    Nehme \( \delta = \frac{1}{n} \Rightarrow \exists 
    x_n \in X: d(x_n, x) < \frac{1}{n} \) und 
    \( d(f(x_n), f(x_0)) \geq \varepsilon > 0 \).
    Beachte: \(x_n \rightarrow x_0\) und 
    \(f(x_n) \not \rightarrow f(x_0) \).
\end{bew}
\end{document}