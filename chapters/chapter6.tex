\documentclass[../ana2.tex]{subfiles}
\begin{document}
\setcounter{section}{5}
\section{Integrationstechniken}
Sei \( \abb{f}{I}{\C} \) Regelfunktion \( a \in I\)
\( \oversett{HDI}{\Rightarrow} F(x) = \integralx{f(t)}{a}{x}{t} \)
ist Stammfunktion von \(f\) (bis auf abzählbare Menge 
\( A \subset I \) ist \(F\) differenzierbar mit 
\( F' = f \) auf \( I \setminus A \)).
\begin{bem}
    Sind \( f_1, f_2 : I \rightarrow \C \) Regelfunktionen
    und \(f_1 = f_2\) auf \( I \setminus A, A \subset I \) 
    abzählbar.
    \[ \Rightarrow \integral{f_1}{a}{b} = \integral{f_2}{a}{b},
    \; \forall a < b \in I \]
\end{bem}
\begin{defi}[Fast überall differenzierbar]
    Eine Funktion \( \abb{F}{I}{\C} \) ist fast überall 
    stetig differenzierbar, falls es Stammfunktion einer 
    Regelfunktion \( \abb{f}{I}{\C} \) ist.
\end{defi}
\begin{notation}
    Wir schreiben \( F = \int f \dx \).
    \[ \integral{f}{a}{b} = F(b) - F(a) \;\forall a,b\in I. \]
\end{notation}
\subsection*{Partielle Integration}
\begin{satz}
    Sind \(\abb{u,v}{I}{\C}\) fast überall stetig 
    differenzierbar, dann ist auch 
    \( u \cdot v \) fast überall stetig differenzierbar und 
    es gilt 
    \[ \int u v' \dx = uv - \int u' v \dx. \]
    Genauer: 
    \[ \integral{uv'}{a}{b} 
    = \left[ uv \right]_a^b - \integral{u'v}{a}{b} 
    \;\forall a,b\in I. \]
\end{satz}
\begin{bew}
\( \abb{u',v'}{I}{\C} \) sind Regelfunktionen und es gibt 
abzählbare Mengen \(A,B\) mit 
\begin{align*}
    & u' = \text{ Ableitung von } u \text{ auf } I \setminus A \\
    & v' = \text{ Ableitung von } v \text{ auf } I \setminus B 
\end{align*}

\(D = A \cup B \) ist abzählbar
\(\Rightarrow\)
\begin{align*}
    &u' = \text{ Ableitung von } u \text{ auf } I \setminus D
    &v' = \text{ Ableitung von } v \text{ auf } I \setminus D
\end{align*}

\( \Rightarrow (uv)' \) ist differenzierbar auf 
\( I \setminus D \) und \( (uv)' = u'v + uv' \)
\[ \Rightarrow \integral{(uv)'}{a}{b} 
= [uv]_a^b = u(b)v(b) - u(a)v(a) \]
\[ \integral{(u'v + uv')}{a}{b} 
= \integral{u'v}{a}{b} + \integral{uv'}{a}{b}. \]
\[ \Rightarrow \integral{uv'}{a}{b} 
= [uv]_a^b - \integral{u'v}{a}{b}. \]
\end{bew}
\begin{bspe}
    \begin{enumerate}
        \item Sei \( a \neq -1, 
        \int x^a \dx = \frac{1}{a+1}x^{a+1} \).
        \item \(a \neq -1\)
        \[ \int \underbrace{x^a}_{u'} \underbrace{\ln x}_{v} \dx 
        = \frac{1}{a+1}x^{a+1} \ln x 
        - \frac{1}{a+1} 
        \underbrace{\int \overbrace{x^{a+1}}^{x^a} \cdot \frac{1}{x} dx}_
    {\frac{1}{a+1}x^{a+1}} \]
        \[ \frac{1}{a+1} x^{a+1} \ln x 
        - \frac{1}{(a+1)^2} x^{a+1} 
        = \frac{x^{a+1}}{(a+1)^2} 
        ((a+1) \ln x + 1). \]
        \item Mit \( v' = 1 \) folgt 
        \[ \int \sqrt{1-x^2} dx 
        = \frac{1}{2} (x \sqrt{1 - x^2} + \arcsin x) \]
        auf \( [-1,1] \).\\
        \[ \int \sqrt{1 + x^2} dx 
        = \frac{1}{2} (x \sqrt{1 + x^2} + \arsinh x) \]
        \[ \int \sqrt{x^2 - 1} dx 
        = \frac{1}{2} (x \sqrt{x^2 - 1} - \arcosh x) \]
        auf \( [1,\infty) \).
    \end{enumerate}
\end{bspe}
\begin{satz}[Substitutionsregel]
    Sei \( \abb{f}{I}{\C} \) Regelfunktion    
    und Stammfunktion \(F\) zu \(f\).\\
    Weiter sei \( \abb{t}{[a,b]}{I} \) 
    stetig differenzierbar und monoton.\\
    Dann gilt:    
    \( F \circ t \) ist Stammfunktion von 
    \( (f\circ t) \cdot t' \)
    und es gilt 
    \[ \integral{f(t(x))}{a}{b} 
    = \integralx{f(t)}{t(a)}{t(b)}{t}. \]    
\end{satz}
\begin{bew}
    1. Behauptung folgt aus der Kettenregel.

    Aus dem HDI folgt dann
    \begin{align*}
        \integral{f(t(x))}{a}{b} 
        &= (F \circ t)(b) - (F \circ t)(a) \\
        &= F(t(b)) - F(t(a))  \\
        &= \integralx{f(t)}{t(a)}{t(b)}{t}.
    \end{align*}
\end{bew}
\begin{bspe}
    \begin{enumerate}\leavevmode
        \item \[ \integral{f(\underbrace{x+c}_{t(x)})}{a}{b} 
        = \integralx{f(t)}{a+c}{b+c}{t}. \]
        \item Ist \(c \neq 0\): 
        \[ \integral{f(\underbrace{cx}_{t(x)})}{a}{b}
        = \frac{1}{c} \integralx{f(t)}{ca}{cb}{t}. \]
        \item 
        \[ \int \frac{t'(x)}{t(x)} \dx 
        = \int f(t) \; dt = \ln \abs{t}. \]
        \[ f(t) = \frac{1}{t}. \]
    \end{enumerate}
\end{bspe}
Anwendung:
Fläche des Einheitskreises:\\
Bild:
%Bild Einheitskreis
\[ I(x) = 2 \integralx{\sqrt{1-t^2}}{x}{1}{t} \]
\[ \oversett{Bsp. 3}{=} 
\left[ t \sqrt{1 - t^2} + \arcsin t \right]_x^1 
= 2(- x\sqrt{1 - x^2} + \arcsin 1 - \arcsin x). \]
\[ I(-1) = 2(\arcsin 1 - \arcsin(-1)) = \pi. \]
\end{document}