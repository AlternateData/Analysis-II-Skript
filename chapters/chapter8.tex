\documentclass[../ana2.tex]{subfiles}

\begin{document}
\setcounter{section}{7}
\section{Uneigentliche Integrale und Reihen}
Bisland \( [a,b]  \) z.B. \( \abb{f}{[1,2]}{\R}, \;
x \mapsto \frac{1}{x} \) \\
\( \abb{f}{(0,2]}{\R} x \mapsto \frac{1}{x} \)
Frage: Wie definiert man Integral über \( [a,b), (a,b] \)
oder \( (a,b) \)? (\( -\infty \leq a < b \leq \infty \))
\begin{defi}
    Sei \(f\) eine Regelfunktion auf dem Intervall \(I\) 
    mit Randpunkten 
    \( a, b, -\infty \leq a < b \leq \infty \).
    \begin{enumerate}
        \item Ist \( I=[a,b) \), so definiert man im 
        Fall der Konvergenz:
        \[ \integral{f(x)}{a}{b} 
        = \underset{[a,b)}{\int} f(x) \dx
        := \limesx{\beta}{b-} \integral{f(x)}{a}{\beta} \]
        In diesem Fall heißt das uneigentliche Integral 
        \( \integral{f(x)}{a}{b} \) (genauer 
        \( \underset{[a,b)}{\int} f(x) \dx \)) konvergent 
        und der Grenzwert dessen Wert.    
        \item Analog im Fall \( I = (a, b], a,b \in \R \).
        \item Ist \( I =(a,b) \) so definiert man
        \[ \integral{f(x)}{a}{b} 
        = \underset{(a,b)}{\int} f(x) \dx 
        := \integral{f(x)}{a}{c} 
        + \integral{f(x)}{c}{b}, \]
        falls die uneigentlichen Integrale auf der rechten
        Seite für ein \(a < c < b\) existieren (und damit
        für alle \( a < c < b \)) und denselben Wert haben.
        \item Ein uneigentliches Integral über \(f\) 
        heißt absolut konvergent, falls das Integral 
        über \(f\) konvergiert.
    \end{enumerate}
\end{defi}
\begin{bem}
    Ist \( f \) eine Regelfunktion auf \( [a,b] \), 
    so stimmt das \gqq{alte} Integral über 
    \( [a,b] \) mit dem \gqq{neuen} Integral nach 
    Def. 1 über \( (a,b), [a,b) \) oder \( (a,b] \) 
    überein.
\end{bem}
\begin{bew}
    \[ \underset{[a,b)}{\int} f(x) \dx = \limesx{\beta}{b} 
    \integral{f(x)}{a}{b} \]
    \( \integral{f(x)}{a}{b} \) ist Stammfunktion von \(f\) 
    die stetig in \(\beta\) ist. \\
    Man definiere \(F(\beta) := \integral{f(x)}{a}{b}\)
    \[ \integral{f(x)}{a}{b} = \limesx{\beta}{b} F(\beta)
    = F(\limesx{\beta}{b} \beta) = F(b) 
    = \integral{f(x)}{a}{b} \]    
\end{bew}
\begin{bsp}
    1.
    \[ \integral{\frac{1}{x^s}}{1}{\infty} 
    \text{ existiert } \Leftrightarrow s > 1 \]
    mit Wert \( \frac{1}{s-1} \), denn 
    \[ \integral{\frac{1}{x^s}}{1}{\beta} 
    = \frac{1}{1-s} [x^{1-s}]^\beta 
    = \frac{1}{1-s} (\beta^{1-s} - 1)
    = \frac{1}{s-1} (1 - \beta^{1-s}), s \neq 1. \]
    \[ \integral{\frac{1}{x^s}}{1}{\beta} 
    = \begin{cases}
        \frac{1}{s-1}(1 - \beta^{1-s}), &s \neq -1.\\
        \ln \beta, &s = -1.
    \end{cases} \]
    \[ \Rightarrow \limes{\beta} \int_1^\beta \frac{\dx}{x^s} \]
    existiert und hat den Wert 
    \( \frac{1}{s-1} \Leftrightarrow s > 1 \) \\
    2.
    \[ \integral{\frac{1}{x^s}}{0}{1} 
    \text{ existiert } \Leftrightarrow s < 1. \]
    und hat dann den Wert \( \frac{1}{1-s} \), da
    \[ \integral{\frac{1}{x^s}}{\alpha}{1} 
    = \begin{cases}
        \frac{1}{1-s} [x^{1-s}]_\alpha^1, &s \neq 1\\
        -\ln \alpha, &s = 1
    \end{cases}
    = \begin{cases}
        \frac{1}{1-s} (1-\alpha^{1-s}), &s \neq 1\\
        \ln \alpha, &s = 1
    \end{cases} \]    
    Grenzwert
    \[ \limesx{\alpha}{0} \int_\alpha^1 \frac{\dx}{x^s} = \frac{1}{1-s} \]
    existiert \( \Leftrightarrow s < 1 \)
    3.
    \[ \integral{e^{cx}}{0}{\infty}, c\in\C, \Re(c) > 0. \]    
    \[ \integral{e^{-cx}}{0}{\beta} 
    = \frac{1}{c} (1-e^{-cx}) \rightarrow \frac{1}{c}, 
    \beta \rightarrow \infty. \]
    4. 
    \[ \int_{-\infty}^\infty \frac{\dx}{1+x^2} = \pi \]
    beide Grenzen \( -\infty, \infty \) kritisch. \\
    Fall: \(s \neq -1\) überlege man selbst.
\end{bsp}
\begin{satz}[Majorantenkriterium]
    Seien \( f,g \) Regelfunktionen auf 
    \( [a,b), \abs{f} \leq g \). 
    Existiert \( \integral{g(x)}{a}{b} \), so existiert 
    auch \( \integral{f(x)}{a}{b} \), was 
    absolut konvergent ist.
\end{satz}
\begin{bem}
    Analog für \( (a,b], (a,b) \).
\end{bem}
\begin{bew}
    \[ F(u) = \integral{f(x)}{a}{u}, a \leq u < b \]
    \[ G(u) := \integral{g(x)}{a}{u} \]
    \begin{align*}
        \Rightarrow F(u) - F(v) 
        &= \integral{f(x)}{u}{v}  \\
        \Rightarrow \abs{F(u) - F(v)} 
        &\leq \integral{\abs{f(x)}}{\min(u,v)}{\max(u,v)} \\
        &\leq \integral{g(x)}{\min(u,v)}{\max(u,v)} \\
        &= \abs{G(u) - G(v)} \tag{\(*\)}
    \end{align*}
    Wissen \( \limesx{u}{b} G(u) \) existiert.
    Aus \( (*) \) und Cauchykriterium für Konvergenz 
    folgt \( \limesx{u}{b} F(u) \) existiert.
\end{bew}
\begin{bsp}
    Konvergenz von \( \integral{\frac{\sin x}{x}}{0}{\infty} \) \\
    \( \frac{\sin x}{x} \rightarrow 1 (x \rightarrow 0) \)
    nur \( \infty \) ist kritisch.
\end{bsp}
\begin{bew}
    \[ \integral{\frac{\sin x}{x}}{0}{1} \]
    ist okay.
    \( \Rightarrow \) z.z. \( \integral{\frac{\sin x}{x}}{1}{\infty} \)
    existiert.    
    Warnung!
    \[ \integral{\abs{\frac{\sin x}{x}}}{1}{\infty} 
    = \infty. \]
    Sei \( 1 < R < \infty \).
    \[ \integral{\frac{\sin x}{x}}{1}{R} 
    \overundersett{part.}{int.}{=}
    \integral{\sin x \cdot \frac{1}{x}}{1}{R} 
    = [-\cos x \frac{1}{x}]_1^R 
    - \integral{\frac{\cos x}{x^2}}{1}{R} \]
    \[ = \cos(1) - \frac{\cos R}{R} 
    - \underbrace{\integral{\frac{\cos}{x^2}}{1}{R}}
    _{\text{Majorante } \frac{1}{x^2}} \]
    \[ \abs{\frac{\cos x}{x^2}} \leq \frac{1}{x^2} \]
    auf \( [1,\infty) \) ist \( \frac{1}{x^2} \) 
    integrierbar.\\
    Majorantenkriterium 
    \[ \Rightarrow \limes{R} 
    \integral{\frac{\cos x}{x^2}}{1}{R} \]
    existiert.
\end{bew}
\end{document}