%17.10.2018
\documentclass[../ana2.tex]{subfiles}
\begin{document}
\setcounter{section}{0}

\begin{prosa}

\section{Was ist Analysis?}

\begin{description}[style=nextline]
	\item[Mathematik]
		Streng logisches Herleiten neuer Aussagen (aus möglichst wenigen\\ Grundannahmen, sogenannten Axiomen).
	\item[Analysis]
		Aus dem altgriechischen \gqq{Auflösen}. Analysis hat ihre Grundlage in der \gqq{Infinitesimalrechnung} von Leibniz und Newton.
	\item[Zentrale Begriffe]
		Grenzwerte von Folgen und Reihen, Funktionen, Stetigkeit, Differenzierbarkeit, Integrierbarkeit, Differential- und Integralrechnung, Differentialgleichungen
		(Newton, Maxwell, Schrödinger), unendlich dimensionale Räume.
\end{description}

\begin{bspe}\leavevmode
	\begin{enumerate}[(1)]
		\item Summe über den Kehrwert von Zweierpotenzen.
			\begin{alignat*}{3}
							   &&             S &= \frac{1}{2} + \frac{1}{4} + \cdots + \frac{1}{2^{n}} + \cdots \\
				\Longrightarrow&\quad& 		 2S &= 1 + \frac{1}{2} + \cdots + \frac{1}{2^{n-1}} + \cdots \\
				\Longrightarrow&\quad& 	     2S &= 1 + S \\
				\Longrightarrow&&             S &= 1
			\end{alignat*}
			\(S\) entspricht der Wahrscheinlichkeit, dass bei wiederholtem Werfen \\
			einer Münze irgendwann Kopf vorkommt.\\
		\item Vorsicht!
			\begin{alignat*}{3}
							   &&       S &= 1 + 2 + 4 + \cdots \\
				\Longrightarrow&\quad& 2S &= 2 + 4 + 8 + \cdots = S - 1\\
				\Longrightarrow&\quad&  S &= \minus 1
			\end{alignat*}
			Natürlich Quatsch! Formales Rechnen kann gefährlich sein!
	\end{enumerate}
\end{bspe}

\begin{description}
	\item[Fragestellungen in dieser Vorlesung]\leavevmode
		\begin{itemize}[-]
			\item Was sind mathematische Aussagen?
			\item Wie macht man Beweise, wie findet man sie? (learning by doing)
			\item logische Zusammenhänge
			\item Was sind Zahlen?
		\end{itemize} 
\end{description}

\end{prosa}

\end{document}