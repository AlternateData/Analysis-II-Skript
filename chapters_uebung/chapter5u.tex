\documentclass[../ana2u.tex]{subfiles}
\begin{document}
\setcounter{section}{4}
\section{Uneigentliche Integrale}
\begin{defi}[Uneigentliches Integral]
    Sei \( a \in \R, b \in \R \cup \set{\infty}, a < b \) 
    und \( \abb{f}{[a,c]}{\R} \)
    integrierbar für alle \( a < c < b \).
    Ist \(f\) im Intervall \( [a,b) \) unbeschränkt 
    oder \( b = \infty \), so nennen wir das Integral
    \[ \integralx{f(t)}{a}{b}{t} \]
    (an der oberen Grenze) uneigentliches Integral.
    Existiert der Grenzwert
    \[ \limesx{c}{b-} \integralx{f(t)}{a}{c}{t} \]
    so sagt man, das uneigentliche Integral
    \( \integralx{f(t)}{a}{b}{t} \) existiert.
\end{defi}
\begin{bem}\leavevmode
    \begin{enumerate}
        \item Integrale an unterer Grenze uneigentlich sind 
        analog definiert
        \item An beiden Seiten uneigentliche Intergrale 
        zerlegt man in zwei einseitig uneigentliche.
    \end{enumerate}
\end{bem}
Man nent ein uneigentliches Integral absolut konvergent, 
wenn das Integral \( \integralx{\abs{f(t)}}{a}{b}{t} \)
existiert.
\begin{bsp}
    \[ \integral{\frac{1}{x^\alpha}}{0}{1} \] konvergent 
    \( \Leftrightarrow \alpha < 1 \)
    \[ \integral{\frac{1}{x^\beta}}{1}{\infty} \]
    konvergent \( \Leftrightarrow \beta > 1 \).
\end{bsp}
\begin{bsp}
    \[ \integral{\frac{\sin x}{x}}{0}{\infty} = \frac{\pi}{2}, \]
    aber \( \integral{\abs{\frac{\sin x}{x}}}{0}{\infty} \)
    existiert nicht.
\end{bsp}
\begin{bsp}
    \[ \integral{e^{-x^2}}{-\infty}{\infty} = \sqrt{\pi} \]
\end{bsp}
\begin{bem}
    Uneigentliche Integrale sind Grenzwerte 
    \gqq{normaler} Integrale, daher übertragen 
    sich die üblichen Rechenregeln.  
\end{bem}
Betrachten wir die Funktion, \(a \in \R\) 
\[ \abb{F}{[0,\infty)}{\R}, (\C, \ldots) \]
\begin{beh}
    \( \limes{x} F(x) \) existiert \\
    \( \Leftrightarrow 
    \forall \varepsilon > 0 \;\exists R \geq a : 
    \forall x,y \geq R: \abs{F(x) - F(y)} < \varepsilon \).
\end{beh}
\begin{bew}
    \gqq{\( \Rightarrow \)}: Wie bei Folgen. 
    \( L = \limes{x} F(x) \)
    \begin{align*}
        &\Rightarrow \forall \varepsilon > 0 \;\exists R 
        \geq a : \abs{F(x) - L} < \frac{\varepsilon}{2} 
        \; \forall x \geq R \\
        &\Rightarrow \abs{F(x) - F(y)} 
        = \abs{F(x) - L + L - F(y)} \\
        &\leq \abs{F(x) - L} + \abs{L - F(y)} < \varepsilon.
    \end{align*}
    \gqq{\( \Leftarrow \)}:   
    \( a_n = F(n) \) ist eine Folge und 
    \[ \forall \varepsilon > 0 \;\exists R \geq a: 
    \abs{a_n - a_m} < \varepsilon \forall n,m \geq R \in\N \]
    \[ \Rightarrow L = \limes{n} a_n = \limes{n} F(n) \]
    existiert.\\
    Sei \( \varepsilon > 0, \)
    \begin{align*}
        &R_1 \geq a: \abs{F(n) - L} 
        < \frac{\varepsilon}{2} \;\forall n\geq R_1. \\
        &R_2 \geq a : \abs{F(x) - F(y)} < \frac{\varepsilon}{2}
        \;\forall x,y \geq R_2.
   \end{align*}
    Setze \( R := \max\set{R_1, R_2}, x \geq R, 
    k \leq x < k+1, k\in\N \)
    \begin{align*}
        \Rightarrow \abs{F(x) - L} &= \abs{F(x) - F(k+1) 
        + F(k+1) - L} \\
        &\leq \abs{F(x) - F(k+1)} + \abs{F(k+1) - L} \\
        &< \frac{\varepsilon}{2} + \frac{\varepsilon}{2} \\
        &= \varepsilon.
    \end{align*}
\end{bew}
\subsection{Konvergenzkriterien}
\subsubsection{Cauchy-Kriterium}
Sei \(a\in\R, b \in \R \cup \set{a}, a < b\).
\(f\) ist definiert auf \([a,b)\) und integrierbar
in allen Intervallen \( [a,c], a < c < b \).
Dann gilt \( \integralx{f(t)}{a}{b}{t} \) 
konvergiert
\[ \Leftrightarrow \forall \varepsilon > 0 \;\exists 
c \in [a,b) \forall x,y \in [c,b):  \]
\[ \abs{ \integralx{f(t)}{x}{y}{t} } < \varepsilon. \]
\subsubsection{Monotoniekriterium}
Sei \( \abb{f}{[a,b)}{\R_{\geq 0}} \) nicht negativ und in
allen Intervallen \( [a,c] \subset [a,b) \) integrierbar.
Dann konvergiert das Intervall \( \integralx{f(t)}{a}{b}{t} \)
genau dann, wenn die Integrale
\[ \integralx{f(t)}{a}{c}{t}, a \leq c < b \]
beschränkt sind.
\subsubsection{Majorantenkriterium, Minorantenkriterium}
Siehe Vorlesung.
\end{document}