\documentclass[../ana2u.tex]{subfiles}
\begin{document}
\setcounter{section}{0}

\section{Komplexer Logarithmus}

\( z\in\C, a\in\R_+ \)
\[ a^z = e^{z \ln a} \]
Was ist mit \( a \in \C \setminus \R_+ \)?
\( y = \ln x  \) ist die eindeutige Lösung der Gleichung 
\( x=e^y \).\\
Funktioniert hier, weil \(e^x\) injektiv ist, \dphp{}
\[ x_1 \neq x_2 \Rightarrow e^{x_1} \neq e^{x_2}. \]
Aber \( e^z \) für \( z\in\C \) ist 
\( 2\pi i \)-periodisch.
Ähnliche Probleme hat man auch bei der Definition 
der Umkehrfunktion des Sinus.
\begin{defi}[Komplexer Logarithmus]
    Sei \( z \in \C \). Wir nennen \( w \) den 
    Logarithmus von \(z\), wenn \( z = e^w \) ist.\\
    Wir schreiben \( w = \log z \).
\end{defi}
Wir müssen also entsprechendes \(w\) finden.
Sei also \( w = u + vi \) und setzen 
\( z = r e^{\theta i} \). Wir setzen in die Gleichung 
ein 
\[ z = r e^{\theta i} \overset{!}{=} e^u e^{iv} 
= e^{u+vi} = e^w. \]
\[ \Rightarrow e u = r \text{ bzw. } u = \ln r \]
\[ \text{und } v = \theta = \arg z. \]
Somit gilt: \[ w = \ln r + \theta i \]
und somit ist der komplexe Logarithmus einer 
komplexen Zahl \(z\) gegeben durch 
\[ \log z = \ln \abs{z} + i\arg z. \tag{\(*\)} \]
\begin{bem}
    Da \( \arg z \) einer Zahl 
    \( z \in \C \setminus \set{0} \)
    auf Vielfache von \( 2\pi \) 
    bestimmt ist, hat jede komplexe Zahl ungleich 
    Null unendlich viele Logarithmen.
\end{bem}
\begin{bsp}
    \begin{align*}
        \log(1+i) &\overset{(*)}{=} \ln(\sqrt{2}) 
        + (\frac{\pi}{4} + 2k\pi)i, \forall k \in \Z \\
        &= \frac{1}{2} \ln 2 + (1 + 8k) \frac{\pi}{4}i, 
        \forall k \in \Z \\
    \end{align*}
    Die Logarithmen haben alle denselben Realteil, aber 
    der Imaginärteil variiert um Vielfache von \( 2\pi \).\\
    \( \Rightarrow \) der komplexe Logarithmus ist 
    mehrwertig bzw.\ mehrzweigig.
\end{bsp}
\begin{bsp}
    \[ \log (2-3i) = ? \]
    \begin{itemize}
        \item \( \abs{2-3i} = \sqrt{13} \).
        \item 
        \begin{align*}   
            \arg(2-3i) &= \arctan \left(-\frac{3}{2}\right) + 2\pi k \\
            &= -\arctan\left(\frac{3}{2}\right) + 2\pi k \\
            \log(2 - 3i) &= \ln \sqrt{13} 
            + i (2\pi k - \arctan \left(\frac{3}{2}\right))\\
            &= \frac{1}{2} \ln 13 
            + i (2 \pi k - \arctan \left(\frac{3}{2}\right))
        \end{align*}
    \end{itemize}
    
\end{bsp}
\section{Eigenschaft des komplexen Logarithmus}
Einige Eigenschaften sind ungleich als im reellen, aber 
nicht alle.\\
Sei \( z = x + iy = re^{\theta i} \).
\begin{enumerate}
    \item \( e^{\log z} = e^{\ln r + \theta i} 
    = e^{\ln r}e^{\theta i} = re^{\theta i} = z \).
    \item 
        \begin{align*}
            \log e^z &= \log e^{x+iy} = \ln e^x + (y + 2k\pi)i \\
            &= x + yi + 2k \pi i\\
            &= z + 2k \pi i.
        \end{align*}
\end{enumerate}
Sei nun \( z_1 = r e^{\theta i} \), 
\( z_2 = R \cdot e^{\phi i} \).
Sei \( p\in\Z \).
\begin{align*}
    \log(z_1 z_2) &= \log(rR e^{(\theta + \phi)i}) \\
    &\overset{(*)}{=} \ln(rR) + (\theta + \phi + 2\pi p)i \\
    &= (\ln r + \theta i) + (\ln R + \phi i) + 2\pi p i. \tag{\(**\)}
\end{align*}
\[ \log z_1 = \ln r + (\theta + 2\pi n)i. \]
\[ \log z_2 = \ln R + (\phi + 2\pi m)i. \]
\begin{align*}
    \Rightarrow \log (z_1 z_2) &\overset{(**)}{=} 
    (\log z_1 - 2\pi ni) + (\log z_2 - 2\pi m i) + 2p\pi i \\
    &= \log z_1 + \log z_2 + 2(\underbrace{p-n-m}_{= k}) \pi i\\    
    &= \log z_1 + \log z_2 + 2\pi k i.
\end{align*}
Somit existiert in Abhängigkeit von den 
jeweiligen Zweigen der 
Logarithmen von \(z_1\) und \(z_2\) ein Zweig des Logarithmus
von \(z_1 z_2 \), sodass
\[ \log (z_1z_2) = \log z_1 + \log z_2 \]
Aber im Allgemeinen ist dies nicht richtig! \\
Gleiches gilt für
\[ \log \left( \frac{z_1}{z_2} \right)
= \log z_1 - \log z_2 + 2k\pi i, k \in \Z. \]
Ableitung von \( \log z \)?
Wir betrachten 
\[ \ddx{z} \log z 
= \limesx{\Delta z}{0} 
\frac{\log(z+\Delta z) - \log z}{\Delta z}. \]
Damit die Ableitung existiert, muss der 
Grenzwert existieren, eindeutig sein und darf
nicht davon abhängen, wie \( \Delta z \rightarrow 0 \)
geht.\\
Dies ist unmöglich, da wir unendlich viele Möglichkeiten
für \( \log z \) und \( \log (z + \Delta z) \) für 
jeden Wert von \( \Delta z \) haben. \\
Wir müssen \( \log z \) einschränken, um eine Ableitung 
zu definieren. Aus der Definition von \( \arg z \) folgt,
dass der Imaginärteil von \( \log z = \ln \abs{z} + (\arg z)i \)
in jedem Punkt auf der negativen reellen Achse eine
unstetige Funktion ist.\\
Somit können wir \( \log z \) in keinem dieser Punkte 
differenzieren. Nehmen wir also als Definitionsbereich
\[ \abs{z} > 0, -\pi < \arg z < \pi \]
dann ist 
\[ \Log z = \ln r + \theta i, -\pi < \theta < \pi \]
sogar unendlich oft differenzierbar.
Für die Ableitung gilt dann
\[ \ddx{z} \Log z = \frac{1}{z} \]
Durch die Einschränkung auf \( -\pi < \arg z < \pi \)
erhält man den Hauptzweig von \( \log z \).
\begin{bem}
    Für \( x \in \R \) hatten wir \(\log_a x \) als den Logarithmus
    zur Basis \(a \) definiert. \\
    Für den komplexen Logarithmus verwendet man immer Basis \(e\).
    Beachte: \( \log_\varphi z \) bezeichnet die Einschränkung
    des \(\log z \) auf den Zweig \( \arg_\varphi \). 
    \[ \log_\varphi z = \ln \abs{z} + (\arg_\varphi z)i \]
\end{bem}
Nun können wir für \( a \in \C, z \in \C \) das 
folgende definieren:
\[ a^z := e^{z \log a}. \]
\begin{beh}
    \( i^i \in\R \).
\end{beh}
\begin{bew}
    \[ i^i = e^{i \log i} \]
    \[ \log i \overset{*}{=} \ln \abs{i} + i \cdot \arg i \]
    \[ \arg i = (\frac{\pi}{2} + 2k\pi), k \in \Z \]
    \[ \Rightarrow \log i = (\frac{\pi}{2} + 2k\pi)i \]
    \[ \Rightarrow i^i = e^{i \log i} 
    = e^{i \cdot i(\frac{\pi}{2} + 2 \pi k)} 
    = e^{-(\frac{\pi}{2} + 2\pi k)} \in\R, k\in\Z. \]
    \( \leadsto \) Einschränkung auf Hauptzweig ergibt: 
    \[ i^i = e^{-\frac{\pi}{2}} \]
\end{bew}
\end{document}