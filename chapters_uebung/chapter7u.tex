\documentclass[../ana2u.tex]{subfiles}
\begin{document}
\setcounter{section}{6}
\section{Metrik und offene Mengen}
    \begin{defi}
        Sei \(X\) eine nicht leere Menge. Eine
        Abbildung \( \abb{d}{X\times X}{\R} \)
        heißt Metrik oder Abstand auf \(X\), falls
        für alle \( x,y,z \in X \) gilt:
        \begin{description}
            \item[Positivität] \( d(x,y) \geq 0 \)
            \item[Definitheit] \( d(x,y) = 0 
            \Leftrightarrow x = y \)
            \item[Symmetrie] \( d(x,y) = d(y,x) \)
            \item[\( \Delta \)-Ungleichung] \( d(x,z) 
            \leq d(x,y) + d(y,z) \)
        \end{description}
        Ein metrischer Raum ist eine nicht leere Menge \(X\)
        mit Metrik \(d\) auf \(X\).
    \end{defi}
    \begin{bsp}[Eisenbahnmetrik]
        Es sei \(X = \R^2\)
        \[ \abb{d}{X \times X}{\R}, d(x,y) = \begin{cases}
            \norm{x-y}_2 &, \text{ falls } y = tx \text{ für ein } t \in \R \\
            \norm{x}_2 + \norm{y}_2 &, \text{ sonst}
        \end{cases} \]
    \end{bsp}
    \begin{bew}
        Positivität: Offensichtlich gilt \( d(x,y) \geq 0 \), 
        da \(\norm{\cdot}_2\)
        Metrik auf \(X\) ist. \\
        Definitheit: \( d(x,x) = 0 \), siehe oben \\
        Symmetrie: \( d(x,y) = d(y,x) \), siehe oben \\
        \( \Delta \)-Unleichung: \\
        Fall 1:
        Seien \( x \) und \( z \) auf einer Geraden \(g\) 
        durch den Ursprung, \dphp{} \( \exists t \in \R \), 
        sodass \(z=tx\).\\
        Fall 1.1: \( y \) ist auch auf der Geraden.
        \[ d(x,z) = \norm{x-z}_2 \leq \norm{x-y}_2 + \norm{y-z}_2 
        = d(x,y) + d(y,z). \]
        Fall 1.2: \( y \) ist nicht auf \(g\).
        \begin{align*}
            d(x,z) &= \norm{x-z}_2 \\
            &\leq \norm{x}_2 + \norm{-z}_2  \\
            &\leq \norm{x}_2 + 2\norm{y}_2 + \norm{z}_2 \\
            &= \norm{x}_2 + \norm{y}_2 + \norm{y}_2 + \norm{z}_2 \\
            &= d(x,y) + d(y,z).
        \end{align*}
        Fall 2: Seien \( x \) und \(z\) nicht auf einer Geraden 
        durch den Ursprung.\\
        Fall 2.1: \( y \) liegt weder auf Gerade durch Ursprung 
        mit \(x\) noch mit \(z\). Dann ist 
        \begin{align*}
            d(x,z) &= \norm{x}_2 + \norm{y}_2 \\
            &\leq \norm{x}_2 + 2\norm{y}_2 + \norm{z}_2 \\
            &= d(x,y) + d(y,z).
        \end{align*}    
        Fall 2.2/2.3: (i) \(y\) ist auf einer Geraden durch 
        \( x \) und den Ursprung. Dann folgt
        \begin{align*}
            d(x,z) &\leq \norm{x}_2 + \norm{z}_2 \\
            &= \norm{x - y + y}_2 + \norm{z_2} \\
            &\leq \norm{x - y}_2 + \norm{y}_2 + \norm{z}_2  \\
            &= d(x,y) + d(y,z).
        \end{align*}
        (ii) \(y\) ist auf der Gerade durch \(z\) und den Ursprung 
        analog.
        \( \Rightarrow d \) ist eine Metrik auf \(X\).
    \end{bew}
    \begin{defi}
        Sei \((X,d)\) ein metrischer Raum. 
        Ist \( \varepsilon > 0 \) und \( x\in X \), so 
        heißt 
        \[ U_\varepsilon(x) 
        := \set{ y \in X: d(x,y) < \varepsilon } \]
        \( \varepsilon \)-Umgebung von \(x\).
    \end{defi}
    \begin{defi}[\( \varepsilon \)-Umgebung von \(x\)]
        Eine Teilmenge \( U \subset X \) heißt Umgebung 
        von \(x\) und \(x\) heißt innerer Punkt von \(U\), 
        falls \(U\) eine \(\varepsilon \)-Umgebung von \(x\) 
        enthält.
        \[ M^0 := \set{x \; \vert \; 
        x \text{ ist innerer Punkt von } M } \]
        heißt Inneres von \(M\).
        Eine Teilmenge \( M \subset X \) heißt offen genau 
        dann wenn sie nur innere Punkt enthält.
    \end{defi}
    \begin{defi}
        \( M \subset X \) ist genau dann offen, wenn 
        ihr Komplement abgeschlossen ist.\\
        Sei \( x \in X \) und \( M \subset X \). 
        \(x\) heißt Berührpunkt von \(M\) wenn jede 
        \( \varepsilon \)-Umgebung von \(x\) die Menge 
        \(M\) trifft.\\
        \( x \) Berührpunkt von \(M\) 
        \[ \Leftrightarrow \forall \varepsilon > 0: 
        M \cap U_\varepsilon(x) \neq \emptyset. \]
        \[ \overline{M} := \set{ x \;\vert \; 
        x \text{ ist Berührpunkt von }M } \] 
        heißt abgeschlossene Hülle von \(M\).\\
        \( M \) heißt abgeschlossen, wenn sie alle 
        ihre Berührpunkte enthält.\\
        \(x\) heißt Randpunkt von \( M \) 
        \[ \Leftrightarrow \forall \varepsilon > 0 : 
        M \cap U_\varepsilon(x) \neq \emptyset \] 
        und \[ (x \setminus M) \cap U_\varepsilon(x) \neq \emptyset \]
        \( \delta M := \set{ x \;\vert\; x 
        \text{ ist Randpunkt von } M } \) heißt 
        Rand von \(M\).
    \end{defi}
    \begin{defi}[offene Überdeckung]
        Sei \((X, d)\) ein metrischer Raum und \(M\) eine Teilmeine von
        \(X\). Ein System \(U = \set{U_j, j \in J}, J \subset \N\)
        von offenen Mengen \(U_j\) heißt offene Überdeckung von \(M\),
        wenn
        \[ M \subset \bigcup_{j \in J} U_j \]
        ist.        
    \end{defi}
    \begin{defi}[Überdeckungskompakt]
        Eine Teilmenge \( M \subset X \) eines metrischen
        Raums \( (X,d) \) heißt überdeckungskompakt, 
        wenn jede offene Überdeckung von \(M\) eine 
        endliche Teilüberdeckung enthält.
    \end{defi}
    \begin{beh}
        Überdeckungskompakte metrische Räume sind
        folgenkompakt.
    \end{beh}
    \begin{bew}
        Sei \(X\) ein überdeckungskompakter metrischer Raum
        und \( (x_n)_n \) eine Folge in \(X\).\\
        Angenommen, die Folge \( (x_n)_n \) besitzt 
        keinen Häufungswert in \( X \). Dann existiert für alle 
        \( x \in X \) ein \( \varepsilon_x > 0 \), sodass 
        \( B_{\varepsilon_x}(x) \) höchstens endlich viele 
        Elemente von \( (x_n)_n \) enthält.\\
        Da \( X \) überdeckungskompakt ist und 
        \(B_\varepsilon(x), x \in X\) eine offene Überdeckung
        von \(X\) ist, existiert eine endliche
        Teilüberdeckung \( B_{\varepsilon_{y_1}}, 
        B_{\varepsilon_{y_2}}, \ldots, 
        B_{\varepsilon_{y_n}} \), sodass 
        \( X \subset \bigcup_{j=1}^n B_{\varepsilon_{y_j}}(y_j) \)        
        Somit liegen in \( \bigcup_{j=1}^n 
        B_{\varepsilon_{y_j}} \) auch nur endlich 
        viele Elemente der Folge \( (x_n)_n \). \Lightning{}\\
        \( (x_n)_n \) war keine Folge in \(X\).\\
        Also besitzt \( (x_n)_n \) einen 
        Häufungswert in \(X\) und somit ist \( X \) 
        per Definition folgenkompakt.
    \end{bew}
\subsection{Eigenschaften kompakter metrischer Räume}
    \begin{enumerate}
        \item Jede kompakte Teilmenge 
        eines metrischen Raumes ist 
        beschränkt, abgeschlossen und 
        vollständig.
        \item Im \(\R^n\) ist jede abgeschlossene und 
        beschränkte Menge auch kompakt
        (Satz von Heine-Borel).
        \item Eine abgeschlossene Teilmenge 
        eines kompakten Raums ist kompakt.
        \item Eine Menge mit endlich 
        vielen Elementen ist kompakt.
        \item Stetige Bilder von kompakten Mengen sind
        kompakt.
        \item Kompakte Teilmengen metrischer Räume besitzen
        eine abzählbare\\dichte Teilmenge. Man sagt sie
        sind seperabel.
    \end{enumerate}
    \begin{beh}
        Folgenkompakte metrische Räume sind 
        seperabel.
    \end{beh}
    \begin{bew}
        Sei \(X\) ein folgenkompakter metrischer Raum.\\
        Zunächst zeigen wir, dass es zu jedem 
        \(\varepsilon > 0\) eindlich viele 
        \( \varepsilon \)-Umgebungen gibt, die \(X\) 
        überdecken. \\
        Angenommen es ist \(\varepsilon > 0\) derart, dass
        es keine endliche Menge von \(\varepsilon\)-Kugeln
        gibt, die \(X\) überdeckt.\\
        Dann würde es eine Folge \( (x_k)_k \) in \(X\) 
        geben, sodass 
        \[ x_{k+1} \notin \bigcup_{j=1}^k U_\varepsilon(x_j) \]
        Für die Folgenglieder gilt dann 
        \( d(x_k, x_j) \geq \varepsilon \) für 
        \( k \neq j \).
        Eine solche Folge kann keine konvergente 
        Teilfolge in \(X\) besitzen. \Lightning{} \\
        Also gibt es zu jedem \(k \in \N\) eine endliche 
        Menge \(E_k \subset X\) von Punkten aus \(X\),
        deren \(\frac{1}{k}\)-Umgebung
        ganz \( X \) überdeckt.\\
        Damit ist \( D := \bigcup_k E_k \) eine abzählbare 
        dichte Teilmenge von \(X\).
    \end{bew}
    \begin{bem}
        Mit Hilfe dieses Resultats kann man nun auch
        beweisen, dass folgenkompakte metrische Räume
        überdeckungskompakt sind. \\
        Dies gilt nicht in allgemeinen topologischen 
        Räumen.
    \end{bem}
    \begin{bsp}
        Die abgeschlossene Einheitskugel im \( \R^n \)
        \[ B = \set{x \in \R^n \;\vert\; 
        \norm{x}_2 \leq 1} \]
        ist kompakt (mit Heine-Borel).
    \end{bsp}
    \begin{bsp}
        Sei
        \[ l^2 := \set{(x_i)_i: \sum_{i=1}^\infty \abs{x_i}^2 < \infty} \]
        der Raum der reellen bzw. komplexen Folgen 
        mit konvergenter Quadratsumme. \\
        Es gilt für \( l^2 \):
        \[ \scalarprod{(x_i)_i}{(y_i)_i} 
        = \sum_{i=1}^\infty x_i \overline{y_i} \]
        ist inneres Produkt auf \(l^2\).
        \begin{align*}
            & \norm{\cdot} = \sqrt{\scalarprod{\cdot}{\cdot}} 
            \text{ definiert die Norm auf } l^2 \\
            & \norm{x-y}_{l^2} = d(x,y) \text{ definiert die Metrik auf } 
            l^2
        \end{align*}
        \( \leadsto l^2 \) ist ein metrischer Raum.
        Die Einheitskugel in \( l^2 \) ist gegeben 
        durch 
        \[ C = \set{x \in l^2: \norm{x}_{l^2} \leq 1}. \]
        Die Menge \( C \) ist abgeschlossen und 
        beschränkt, aber sie ist nicht kompakt.
        Um dies zu zeigen, zeigen wir, dass \(C\) nicht 
        folgenkompakt ist.
        Sei \( (e_k)_k \) gegeben durch 
        \( e_k = (0,\dots,0,
        \underbrace{1}_{k\text{-te Stelle}},0,\dots) \). \\
        Dann gilt: 
        \( \sum_{i=1}^\infty \abs{e_k^i}^2 = 1 < \infty \) \\
        \( \Rightarrow e_k \in C \subset l^2 \; \forall k \in \N \)
        \[ \norm{e_k - e_m}_{l^2}^2 
        = \norm{(0, \ldots, 0, 1, 0, 
        \ldots, 0, -1, 0, \ldots)}_{l^2}^2 
        = 2. \]
        Somit kann \( (e_k)_k \subset l^2 \) keine konvergente Teilfolge 
        in \( l^2 \) besitzen. \\
        \( \Rightarrow C \) ist nicht folgenkompakt.
        Somit ist \( C \) nicht kompakt.
    \end{bsp}
    
\end{document}